\documentclass[11pt,a4paper]{article}
\usepackage[utf8]{inputenc}
\usepackage[brazil]{babel}
\usepackage{geometry}
\geometry{a4paper, margin=2cm}
\usepackage{setspace}
\usepackage{fancyhdr}  % Para rodapé personalizado
\usepackage[hyphens]{url}
\usepackage[colorlinks=true,linkcolor=blue,urlcolor=blue]{hyperref} % Colocar no preâmbulo
\onehalfspacing

\pagestyle{fancy}
\fancyhf{} % limpa cabeçalho e rodapé padrão
\fancyfoot[l]{\tiny \it \\Costa, J.R}


\begin{document}
	
	\begin{center}
		\Large \textbf{Colegiado de Curso no IFCE} \\		 
	\end{center}

\begin{flushright}
	\scriptsize \it Prof. Jonatha Costa\footnote{\it Coordenação do Curso de Bacharelado em Engenharia de Controle e Automação (CBECA), \today.}
\end{flushright}

	
	\vspace{0.5cm}
	
	O \textbf{\textbf{colegiado} do curso} é o órgão \textbf{colegiado} local, de caráter consultivo e deliberativo, cuja função é assegurar a qualidade acadêmica e pedagógica, articulando as necessidades do corpo discente, docente e técnico-administrativo às diretrizes institucionais e ao Projeto Pedagógico de Curso (PPC).  
	
	No \textit{Regulamento da Organização Didática do IFCE} (Resolução CONSUP nº 35/2015), a importância do colegiado é evidenciada em cinco artigos fundamentais:
	
	% § : parágrafo
	% IV :incivo
	\begin{enumerate}
		\item \textbf{Art. 23, §4º}: estabelece que as solicitações de reestruturação dos Projetos Pedagógicos de Curso (PPC) devem, preferencialmente, ser propostas pelo \textbf{colegiado} de curso, quando houver, ou pela Coordenação Técnico-Pedagógica. O \textbf{colegiado} assume, portanto, a responsabilidade de elaborar parecer técnico-pedagógico fundamentado, conduzindo as alterações de matriz curricular com transparência e participação.
		
		\item \textbf{Art. 36, inciso IV}: prevê a obrigatoriedade das reuniões \underline{bimestrais} dos \textbf{colegiado}s de curso. Essas reuniões, de natureza deliberativa, devem tratar de pautas pedagógicas centrais, como indicadores de desempenho, acompanhamento de planos de ensino, medidas de recuperação e avaliação de alterações curriculares, com registro em ata e divulgação ao corpo acadêmico.
		
		\item \textbf{Art. 109, §6º}: define que, em casos de reprovação por excesso de faltas ou de aprovação por média, a deliberação pode ser atribuída ao \textbf{colegiado} de curso, entre outras instâncias. Nessa função, o \textbf{colegiado} atua como garantidor do devido processo acadêmico, assegurando direito de defesa ao estudante, fundamentação das decisões e registro formal em ata.
		
		\item \textbf{Art. 173, inciso V}: assegura ao estudante o direito de ter representação no \textbf{colegiado} de curso (ou conselho de classe), quando houver. Essa prerrogativa reforça o caráter democrático do \textbf{colegiado}, garantindo voz e voto discente nas decisões pedagógicas, fortalecendo a legitimidade e transparência dos processos decisórios.
		

	\end{enumerate}
	
	Dessa forma, o \textbf{colegiado} do curso consolida-se como instância essencial da governança acadêmica, garantindo que as decisões sobre currículo, avaliação, recursos pedagógicos e situações excepcionais sejam tomadas de forma coletiva, fundamentada e transparente. Ao integrar docentes, discentes e técnicos-administrativos, o \textbf{colegiado} promove equilíbrio entre tradição acadêmica e inovação, preservando a qualidade e a legitimidade do processo formativo.
	
	\vspace{0.5cm}
	
\noindent \textbf{Referência} \\
INSTITUTO FEDERAL DE EDUCAÇÃO, CIÊNCIA E TECNOLOGIA DO CEARÁ. \textit{Regulamento da Organização Didática (ROD)}. Resolução CONSUP nº 35, de 22 de junho de 2015. Fortaleza: IFCE, 2015. Disponível em: 
\url{https://portal.ifce.edu.br/institucional/ensino/regulamento-de-organizacao-didatica-rod}. Acessado em: 18 set. 2025.

	
\end{document}
