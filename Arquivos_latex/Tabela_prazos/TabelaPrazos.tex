\documentclass[authoryear]{elsarticle}

%---------------- Pacotes ----------------%
\usepackage[utf8]{inputenc}  % Codificação UTF-8
\usepackage{array}           % Melhor controle de colunas em tabelas
\usepackage{geometry}        % Ajuste de margens
\usepackage{longtable}       % Tabelas longas
\usepackage{graphicx}        % Inclusão de imagens
\usepackage{hyperref}        % Links clicáveis

%---------------- Configuração da página ----------------%
\geometry{a4paper, margin=1.5cm}

%---------------- Rodapé personalizado ----------------%
% Substitui o rodapé padrão "Submitted to Elsevier" por texto próprio
\makeatletter
\def\ps@pprintTitle{%
	\let\@oddhead\@empty
	\let\@evenhead\@empty
	\def\@oddfoot{\footnotesize \textit{Submitted by Prof. Jonatha Costa, CCECA} \hspace{10cm} \textit{\today}\hfill}%
	\let\@evenfoot\@oddfoot
}
\makeatother

\begin{document}
	
	\begin{frontmatter}
		\title{IFCE \textit{campus Macaranaú}\\Regulamento da Organização Didática (ROD)\\
			Tabela Resumo de processos com requisitos e prazos}
		
		\begin{abstract}
			Este documento apresenta um conjunto de tabelas resumidas e detalhadas referentes ao Regulamento da Organização Didática (ROD) de cursos superiores, com ênfase nas normas de matrícula, reingresso, trancamento de disciplinas, aproveitamento de componentes curriculares e validação de conhecimentos. As tabelas foram estruturadas para fornecer consulta rápida e detalhada, permitindo que estudantes, docentes e gestores acadêmicos identifiquem prazos, requisitos e artigos específicos do regulamento de forma eficiente. Links diretos para o ROD completo, hospedado online, facilitam a verificação das seções e artigos correspondentes. Este trabalho visa aprimorar a organização acadêmica e promover maior transparência e acessibilidade às normas institucionais.
		\end{abstract}
		
	\end{frontmatter}
	
\section*{Introdução}
Este documento apresenta um resumo do Regulamento da Organização Didática (ROD) dos cursos superiores, especialmente útil para a Coordenação do Curso de Engenharia de Controle e Automação (CCECA), bem como para corpo docente e discente. Na Tabela \ref{tab:prazos_rod} são apresentados itens como matrícula, reingresso, trancamento de disciplinas, aproveitamento e validação de componentes curriculares, facilitando consulta rápida às normas institucionais.


\section*{Tabela Completa – Detalhada}
\scriptsize

\begin{longtable}{|c|p{3cm}|p{3cm}|p{3.5cm}|p{3.5cm}|p{2cm}|}  
	\caption{Resumo de normas do ROD\label{tab:prazos_rod}} \\
	\hline 
	\textbf{Id} & \textbf{Tema} & \textbf{Descrição} & \textbf{Prazo} & \textbf{Observações/Requisitos} & \textbf{ROD (Seção/Artigo)} \\
	\hline
	1.0 & Aproveitamento de Componentes Curriculares & Disciplinas cursadas em outras instituições podem ser aproveitadas. & \textit{Após início do semestre}: Ingressantes: até 10 dias Veteranos: até 30 dias . & Exige $\geq$75\% da carga horária e conteúdo compatível. Proibido para estágio, TCC e atividades complementares. & \href{https://github.com/JonathaCosta-IA/CCECA/blob/main/02_Rod_atualizado1.pdf}{Cap. IV – Seção I – Art. 130--133} \\
	\hline
	1.1 & Revisão de Aproveitamento & Contestação do resultado da análise. & Até 5 dias após divulgação. & Analisada por dois professores designados. & \href{https://github.com/JonathaCosta-IA/CCECA/blob/main/02_Rod_atualizado1.pdf}{Cap. IV – Seção I – Art. 135--136} \\
	\hline
	1.2 & Conclusão do Processo de Aproveitamento & Finalização do trâmite de aproveitamento. & Até 30 dias após solicitação inicial. & Inclui análise e eventual revisão. & \href{https://github.com/JonathaCosta-IA/CCECA/blob/main/02_Rod_atualizado1.pdf}{Cap. IV – Seção I – Art. 136} \\
	\hline
	2.0 & Validação de Conhecimentos & Reconhecimento de saberes prévios acadêmicos ou profissionais. & Solicitação nos 30 primeiros dias do semestre. Conclusão em até 50 dias letivos. & Avaliação por banca de 2 docentes. Nota mínima: 7,0 (graduação). & \href{https://github.com/JonathaCosta-IA/CCECA/blob/main/02_Rod_atualizado1.pdf}{Cap. IV – Seção II – Art. 137--145} \\
	\hline
	2.1 & Revisão da Validação & Contestação do resultado de validação. & Até 2 dias após resultado. & Dois professores designados em nova banca. & \href{https://github.com/JonathaCosta-IA/CCECA/blob/main/02_Rod_atualizado1.pdf}{Cap. IV – Seção II – Art. 145} \\
	\hline
	2.2 & Aproveitamento Extraordinário & Abreviação de curso por desempenho excepcional. & Conforme calendário definido. & Provas específicas por banca especial. & \href{https://github.com/JonathaCosta-IA/CCECA/blob/main/02_Rod_atualizado1.pdf}{Cap. IV – Seção III – Art. 146} \\
	\hline
	3.0 & Matrícula Inicial & Ingresso após processo seletivo. & Conforme edital. & Obrigatório cursar todos os componentes do 1º semestre. A partir do 2º: mínimo 12 créditos. & \href{https://github.com/JonathaCosta-IA/CCECA/blob/main/02_Rod_atualizado1.pdf}{Cap. II – Seção I – Art. 75--78} \\
	\hline
	3.1 & Renovação de Matrícula & Confirmação de vínculo acadêmico. & Definido em calendário. Regularização até 5 dias após prazo online. & Se não renovada, caracteriza abandono. & \href{https://github.com/JonathaCosta-IA/CCECA/blob/main/02_Rod_atualizado1.pdf}{Cap. II – Seção II – Art. 79--81} \\
	\hline
	3.2 & Matrícula Especial & Permite cursar disciplinas isoladas. & Solicitação nos 50 primeiros dias do período anterior. & Máx. 3 disciplinas (salvo revalidação de diploma estrangeiro). Não gera vínculo. & \href{https://github.com/JonathaCosta-IA/CCECA/blob/main/02_Rod_atualizado1.pdf}{Cap. I – Seção III – Art. 63--69} \\
	\hline
	4.0 & Reingresso & Retorno após abandono. & Até 5 anos após saída. & Exige vaga e nada consta da biblioteca. Proibido se abandono ocorreu no 1º período/ano. & \href{https://github.com/JonathaCosta-IA/CCECA/blob/main/02_Rod_atualizado1.pdf}{Cap. I – Seção IV – Art. 70--72} \\
	\hline
	4.1 & Trancamento de Matrícula & Suspensão temporária dos estudos. & A qualquer tempo, exceto no 1º período (salvo exceções). & Máx. 4 semestres. Necessário nada consta da biblioteca. Exceções: doença, gravidez de risco, serviço militar. & \href{https://github.com/JonathaCosta-IA/CCECA/blob/main/02_Rod_atualizado1.pdf}{Cap. V – Seção I – Art. 149--154} \\
	\hline
	4.2 & Trancamento de Componente Curricular & Trancar disciplina específica. & Até 30 dias após início do período letivo. & Permitido apenas se mantiver $\geq$12 créditos. Proibido no 1º período. & \href{https://github.com/JonathaCosta-IA/CCECA/blob/main/02_Rod_atualizado1.pdf}{Cap. V – Seção II – Art. 155} \\
	\hline
	5.0 & Avaliação de Aprendizagem & Prazos de correção e revisão de avaliações. & Correção: até 10 dias. Pedido de revisão: até 2 dias após resultado. & Nota mínima: 7,0 para aprovação em graduação. & \href{https://github.com/JonathaCosta-IA/CCECA/blob/main/02_Rod_atualizado1.pdf}{Cap. III – Seção I – Art. 94--100} \\
	\hline
	5.1 & Segunda Chamada de Avaliação & Possibilidade de realizar prova substitutiva em caso de falta justificada. & Solicitação em até 5 dias úteis após a avaliação perdida. & Exige comprovação documental da justificativa (doença, atividade institucional etc.). & \href{https://github.com/JonathaCosta-IA/CCECA/blob/main/02_Rod_atualizado1.pdf}{Cap. III – Seção IV – Art. 103--107} \\
	\hline
	6.0 & Colação de Grau & Formalização da conclusão. & Após integralização da matriz curricular. & Exige cumprimento de TCC, estágio, atividades complementares e ENADE. & \href{https://github.com/JonathaCosta-IA/CCECA/blob/main/02_Rod_atualizado1.pdf}{Cap. VI – Seção V--VI – Art. 166--168} \\
	\hline
	
7.0 & Diploma & Emissão e registro do diploma após colação de grau. & Até 90 dias após a colação. & Exige integralização curricular, nada consta e assinatura do egresso. Diploma registrado em livro próprio e expedido pela Reitoria. & \href{https://github.com/JonathaCosta-IA/CCECA/blob/main/02_Rod_atualizado1.pdf}{Cap. VI – Seção V--VI – Art. 166--168} \\
		\hline
		
		
		
	\multicolumn{6}{c}{\scriptsize Fonte: Autor, 2025} \\
\end{longtable}

		
	\section*{Referência}
	Regulamento da Organização Didática (ROD). Disponível em: 
	\url{https://github.com/JonathaCosta-IA/CCECA/blob/main/02_Rod_atualizado1.pdf}. Acessado em: 05/09/2025.
	
\end{document}
